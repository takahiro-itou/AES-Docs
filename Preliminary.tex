
\section{準備}

\subsection{$\mathbb{F}_2$ 上の既約多項式}

\begin{proposition}
$\mathbb{F}_2$ 上の 4 次以下の既約多項式は次の 8 個。
1 次式 : $x$, $x+1$,
2 次式 : $x^2+x+1$,
3 次式 : $x^3+x+1$, $x^3+x^2+1$,
4 次式 : $x^4+x+1$, $x^4+x^3+1$, $x^4+x^3+x^2+x+1$.
\end{proposition}

\begin{proof}
1 次式 $x$ および $x+1$ は明らかに既約。
また、2 次以上で定数項が $0$ のものは、
$x$ で割り切れるので、既約ではない。
よって、以後定数項が $1$ のものだけ考えれば良い。


2 次式は $x^1$ と $x^2+x+1$ の 2 個だけであるが、
前者は $x^2+1 = (x+1)^2$ となり既約ではない。
一方後者は $x^2+x+1 = x(x+1) + 1$ となり、
これは $x$ でも $x+1$ でも割り切れないので既約である。

3 次式は $x^3+1$, $x^3+x+1$, $x^3+x^2+1$, $x^3+x^2+x+1$ の 4 個ある。
このうち、
$x^3+1 = (x+1)(x^2+x+1)$ と
$x^3+x^2+x+1 = (x+1)(x^2+1)$ は既約ではない。
一方、3 次式が既約でないなら、1 次式を因子に持つが、
\begin{eqnarray}
x^3 + x + 1 =
\begin{cases}
x (x^2+1) + 1 & \\
(x+1) (x^2+x) + 1, &
\end{cases}
x^3 + x^2 + 1 =
\begin{cases}
x (x^2 + x) + 1 & \\
(x+1) x^2 + 1, &
\end{cases}
\end{eqnarray}
となって、$x$ でも $x+1$ でも割り切れないので、この 2 個
$x~3 + x + 1$ と $x^3 + x^2 + 1$ は既約である。

4 次式 $P = x^4 + a_3 x^3 + a_2 x^2 + a_1 x + 1$ が既約でないとすると、
(1 次式) $\times$ (3 次式)
か
(2 次式) $\times$ (2 次式)
と書ける。
後者は (2 次式) が既約でなければ
(1 次式) $\times$ (3 次式)
に帰着されるので、
\begin{equation}
(x^2 + x + 1)^2 = x^4 + x^2 + 1
\end{equation}
に限定される。
また、定数項が $1$ なので、
$P = x (x^3 + a_3 x^2 + a_2 x + a_1) + 1$
となって $x$ では割り切れないことは明らかである。
よって $P$ を $x+1$ で割ってみて、
割り切れれば可約、割り切れなければ既約である。
但し $x^4 + x^2 + 1$ は可約だと判明しているので除外する。
実際に割ってみると、
\begin{eqnarray}
x^4 + 1 &=& (x + 1) (x^3 + x^2 + x + 1), \\
x^4 + x^2 + x + 1 &=& (x + 1) (x^3 + x^2 + 1), \\
x^4 + x^3 + x + 1 &=& (x + 1) (x^3 + 1), \\
x^4 + x^3 + x^2 + 1 &=& (x + 1) (x^3 + x + 1)
\end{eqnarray}
は可約であり、一方
\begin{eqnarray}
x^4 + x + 1 &=& (x + 1) (x^3 + x^2 + x) + 1, \\
x^4 + x^3 + 1 &=& (x + 1) x^3 + 1, \\
x^4 + x^3 + x^2 + x + 1 &=& (x + 1) (x^3 + x) + 1
\end{eqnarray}
は既約である。

以上より、$\mathbb{F}_2$ 上の 4 次以下の多項式について、
上記の 8 個は既約多項式であり、それ以外は既約では無いことが示された。
\end{proof}
